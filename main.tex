\documentclass[11pt, oneside]{article}
\usepackage{geometry} 

\geometry{letterpaper}

\usepackage{graphicx}	
\usepackage{amssymb}
\usepackage{amsmath}
\usepackage{fancyhdr}
\usepackage[shortlabels]{enumitem}

% Theorems, definitions, oh my!
\newtheorem{definition}{Definition}
\newtheorem{theorem}{Theorem}

\pagestyle{fancy}

\lhead{Patrick Conrey}
\rhead{10-725 -- Homework 1}
\rfoot{\today}

\begin{document}

\section*{Some Useful Definitions}
\begin{definition}{Fibration:}
A fibration is a mapping between two topological spaces that has the homotopy lifting property for every space $X$.
\end{definition}

\begin{theorem}[Pythagorean theorem]
\label{pythagorean}
This is a theorem about right triangles and can be summarised in the next 
equation 
\[ x^2 + y^2 = z^2 \]
\end{theorem}

\hline

%PROBLEM 1
\section{Section Title}
\subsection{Problem Name}

\end{document}  